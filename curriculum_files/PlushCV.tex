%%%%%%%%%%%%%%%%%%%%%%%%%%%%%%%%%%%%%%%
% PlushCV - One Page Two Column Resume
% LaTeX Template
% Version 1.0 (11/28/2021)
%
% Author:
% Shubham Mazumder (http://mazumder.me)
%
% Hacked together from:
% https://github.com/deedydas/Deedy-Resume
%
% IMPORTANT: THIS TEMPLATE NEEDS TO BE COMPILED WITH XeLaTeX
%
% 
%%%%%%%%%%%%%%%%%%%%%%%%%%%%%%%%%%%%%%
% 
% TODO:
% 1. Figure out a smoother way for the document to flow onto the next page.
% 3. Add more icon options 
% 4. Fix hacky left alignment on contact line
% 5. Remove Hacky fix for awkward extra vertical space
% 
%%%%%%%%%%%%%%%%%%%%%%%%%%%%%%%%%%%%%%
%
% CHANGELOG:
%
%%%%%%%%%%%%%%%%%%%%%%%%%%%%%%%%%%%%%%%
%
% Known Issues:
% 1. Overflows onto second page if any column's contents are more than the vertical limit.
%%%%%%%%%%%%%%%%%%%%%%%%%%%%%%%%%%%%%%
%%Icons:
%%Main: https://icons8.com/icons/carbon-copy
%%%%%%%%%%%%%%%%%%%%%%%%%%%%%%%%%%

\documentclass[]{plushcv}
\usepackage{fancyhdr}
\pagestyle{fancy}
\fancyhf{}
\begin{document}

%%%%%%%%%%%%%%%%%%%%%%%%%%%%%%%%%%%%%%
%
%     TITLE NAME
%
%%%%%%%%%%%%%%%%%%%%%%%%%%%%%%%%%%%%%%

\namesection{Caterina}{Bonan}{Theoretical Linguistics | NLP Engineering}{\contactline{\href{https://orcid.org/0000-0002-4808-6865}{OrcID}}{\href{https://github.com/CaterinaBi}{CaterinaBi}}{\href{https://www.linkedin.com/in/caterinabonan/}{LinkedIn}}{\href{mailto:caterina.bonan@outlook.com}{caterina.bonan@outlook.com}}{\href{tel:+447983818663}{07983818663}}}

% \namesection{Firstname}{Lastname}{Full Stack Software Engineer}{\contactline{\href{https://www.mazumder.me}{mazumder.me}}{\href{https://www.github.com/sansquoi}{sansquoi}}{\href{https://www.linkedin.com/mazumders}{mazumders}}{\href{mailto:shubham.mazumder@gmail.com}{first.last@email.com}}{\href{tel:+1999999999}{9999999999}}}

%%%%%%%%%%%%%%%%%%%%%%%%%%%%%%%%%%%%%%
%
%     COLUMN ONE
%
%%%%%%%%%%%%%%%%%%%%%%%%%%%%%%%%%%%%%%

\begin{minipage}[t]{0.70\textwidth} 


%%%%%%%%%%%%%%%%%%%%%%%%%%%%%%%%%%%%%%
%     EXPERIENCE
%%%%%%%%%%%%%%%%%%%%%%%%%%%%%%%%%%%%%%

\section{Selected work experience}

%\runsubsection{AiCore}
%\descript{| Machine Learning and Data Engineering trainee}
%\location{May 2022 – Present | London, UK}
%\vspace{\topsep} % Hacky fix for awkward extra vertical space
%Training in AI \& Data using industry-standard tools. Working on industry projects throughout.
%\sectionsep
%\sectionsep
%\begin{tightemize}
%\item Skills: Web scraping; Data science (Data cleaning \& visualization), Machine learning (Regression, random forest, SVMs, PCA, t-SNE ), Deep learning (Pytorch, neural networks, CNNs, RNNs, Autoencoders, GANs), NLP (LSTMs, transformers, attention mechanism, BERT, HuggingFace), Training in the cloud (AWS EC2), Deployment (Python Flask, Cron, API creation, AWS Lambda, Docker, ONNX, TensorFlow.JS, GCP AI services).
%\end{tightemize}
%\sectionsep

\runsubsection{University of Cambridge}
\descript{| Postdoctoral Researcher in Theoretical Linguistics}
\location{September 2019 – Current | Cambridge, UK}
% \vspace{\topsep} % Hacky fix for awkward extra vertical space
Research in comparative syntax (data collection, visualization, modelling). Published 8 research articles, 2 edited volumes, 1 research monograph. Awarded a SNSF grant twice.
\sectionsep
\vspace*{2mm}
\begin{tightemize}
\item Tools: Excel, Python, MatPlotLib, Selenium, LaTeX, VS code, HF Transformers.
\item Links to scientific writing samples: \href{https://ebooks.iospress.nl/doi/10.3233/SHTI220702}{\textbf{IOS Press}}, \href{https://github.com/CaterinaBi/health-communication-paper2/blob/main/paper2023/first-submission/BonanSamo2023.pdf}{\textbf{Datasets (pre-print)}}, \href{https://periodicos.unb.br/index.php/cs/article/view/40559/33149}{\textbf{Cadernos}}, \href{https://doi.org/10.16995/glossa.5714}{\textbf{Glossa}}.
\end{tightemize}
\sectionsep

\runsubsection{Université de Genève}
\descript{| Research Assistant in Theoretical Linguistics}
\location{September 2015 – August 2019 | Geneva, CH}
Worked on comparative Romance syntax (data collection, visualization, mmodelling). Published 3 research articles, 1 co-edited volume, and a PhD dissertation.
\vspace*{1mm}
\begin{tightemize}
\item Tools: Excel, Python, Praat, LaTeX, Word and Power Point.
\end{tightemize}
%\sectionsep

%\runsubsection{ELSE: Conseil, Création, Édition}
%\descript{| Translator (IT>FR) }
%\location{March 2014 – May 2014 | Paris, FR}
%Translation of book 'Grands écrivains: Les auteurs célèbres vus par de grands photographes' (Éditions du Chêne, 522 pp.; ISBN 978-2-8123-1085-0).
%\sectionsep

%%%%%%%%%%%%%%%%%%%%%%%%%%%%%%%%%%%%%%
%      Training
%%%%%%%%%%%%%%%%%%%%%%%%%%%%%%%%%%%%%

\vspace*{-2mm}
\section{Relevant trainings}

\runsubsection{AiCore}
\descript{| Machine Learning and Data Engineering trainee}
\location{May 2022 – Present | London, UK}
%\vspace{\topsep} % Hacky fix for awkward extra vertical space
Training in AI \& Data using industry-standard tools. Working on industry projects throughout.
%\sectionsep
%\sectionsep
\vspace*{1mm}
\begin{tightemize}
\item Skills \& tools: Web scraping; Data cleaning \& visualization; Machine learning (Regression, random forest, SVMs, PCA, t-SNE ); Deep learning (Pytorch, neural networks, CNNs, RNNs, Autoencoders, GANs), NLP (LSTMs, transformers, attention mechanism, BERT, HuggingFace); Training in the cloud (AWS EC2); Deployment (Python Flask, Cron, API creation, AWS Lambda, Docker, ONNX, TensorFlow.JS, GCP AI services).
\end{tightemize}
\sectionsep

%%%%%%%%%%%%%%%%%%%%%%%%%%%%%%%%%%%%%%
%      Voluntary
%%%%%%%%%%%%%%%%%%%%%%%%%%%%%%%%%%%%%

\vspace*{-2mm}
\section{Open-source collaborations}

\runsubsection{Polyglot @ EleutherAI}
\descript{| NLP engineer \& Language Expert.}
\location{Repo: \href{https://github.com/EleutherAI/polyglot}{eleutherAI-polyglot}}
\begin{tightemize}
\item Creation of multilingual models that show better non-English performance than mBERT, BLOOM, and XGLM.
\item Currently working towards the establishment of a Romance dataset (3T/language target). 
\end{tightemize}
\sectionsep

\runsubsection{Omdena}
\descript{| NLP engineer \& Chapter Lead.}
\location{Webpage: \href{https://omdena.com/}{Global impact with Artificial intelligence}}
\begin{tightemize}
\item Worked on 4 NLP-powered 'AI for good' cooperative projects (two conversational AI assistants, and two classifiers to prevent online hate speech and child grooming);
\item Currently preparing the first challenge ('standalone chrome extension for fake news detection') for the Cambridge Local Chapter, which I lead.
\end{tightemize}
\sectionsep

%%%%%%%%%%%%%%%%%%%%%%%%%%%%%%%%%%%%%%
%     Projects
%%%%%%%%%%%%%%%%%%%%%%%%%%%%%%%%%%%%%%

\vspace*{-2mm}
\section{Additional ai project experience}

%\runsubsection{POLYGLOT}
%\descript{| Python, Selenium, NLTK, PyTorch.}
%\location{Repo: \href{https://github.com/EleutherAI/polyglot}{eleutherAI-polyglot}}
%\begin{tightemize}
%\item Creation of multilingual models that show better non-English performance than mBERT, BLOOM, and XGLM.
%\item Currently working towards the establishment of a Romance dataset (3T target/language). 
%\end{tightemize}
%%\sectionsep

%\runsubsection{NLP CLASSIFIER TO STOP ONLINE VIOLENCE}
%\descript{| Python, Selenium, PyTorch.}
%\location{Webpage: \href{https://omdena.com/projects/stop-online-violence-against-}{nlp-classifier-omdena-marseille}}
%\begin{tightemize}
%\item Data scraping, data annotation, data pre-processing, ML model creation;
%\item Will be deployed as an API (stand-alone Chrome extension).
%\end{tightemize}
%\sectionsep

%\runsubsection{CONVERSATIONAL AI CHATBOT FOR FINANCIAL SUPPORT}
%\descript{| Python, Selenium, PyTorch.}
%\location{Webpage: \href{https://omdena.com/chapter-challenges/conversational-ai-chatbot-for-people-affected-by-high-inflation-and-increased-cost-of-living/}{nlp-classifier-omdena-marseille}}
%\begin{tightemize}
%\item Data scraping, manual labelling, data pre-processing, algorithm selection/deployment;
%\item Will be deployed as a conversational AI chatbot.
%\end{tightemize}
%\sectionsep

\runsubsection{FB MARKETPLACE RECOMMENDATION RANKING SYSTEM}
\descript{| Python, PyTorch, AWS, EKS.}
\location{Repo: \href{https://github.com/CaterinaBi/aicore-recommendation-ranking-system}{aicore-recommendation-ranking-system}}
\begin{tightemize}
\item Currently training a model that works with images and text to make recommendations to potential buyers based on their demographic information;
\item Will host the app on EKS to orchestrate the containers that maintain/retrain the model.
\end{tightemize}
\sectionsep

\runsubsection{WEB SCRAPING PIPELINE}
\descript{| Python, Selenium, Docker, GitHub Actions.}
\location{Repo: \href{https://github.com/CaterinaBi/aicore-web-scraping-pipeline}{aicore-web-scraping-pipeline}}
\begin{tightemize}
\item Developed a module that scrapes data to upsert information to a database;
\item Pushed an image of the containerised application in a CI/CD pipeline that performs unit testing and deploys the image to an EC2 instance.
\end{tightemize}
\sectionsep

%\runsubsection{CHATBOT WITH DEEP NLP}
%\descript{| Python, Seq2Seq, TensorFlow.}
%\location{Repo: \href{https://github.com/CaterinaBi/find-a-holiday-chatbot}{find-a-holiday-chatbot}}
%\begin{tightemize}
%\item Chatbot that utilises a Seq2Seq architecture and TensorFlow to help choose the perfect holiday destination.
%\end{tightemize}
%\sectionsep

%\runsubsection{IMAGE CLASSIFIER FOR FLOWER RECOGNITION}
%\descript{| Python, TensorFlow.}
%\location{Repo: \href{https://github.com/CaterinaBi/udacity-image-classifier}{udacity-image-classifier}}
%\begin{tightemize}
%\item Python image classification application that runs from the command line. 
%\item The application trains a deep learning model on the 102 Category Flower Dataset, and uses it to classify images of flowers.
%\end{tightemize}
%\sectionsep

\runsubsection{COMPUTER VISION APPLICATION}
\descript{| Python, Teachable Machine, OpenCV}
\location{Repo: \href{https://github.com/CaterinaBi/aicore-computer-vision}{aicore-computer-vision}}
\begin{tightemize}
\item Trained a computer vision model and used Tensorflow to detect whether Rock, Paper or Scissors is shown to the camera in real-time and with a high accuracy
\item Used the OpenCV library to access the webcam and play Rock Paper Scissors with the computer using the image from the camera
\end{tightemize}
%\sectionsep

%\runsubsection{PRE-TRAINED IMAGE CLASSIFIER FOR DOG BREEDS}
%\descript{| Python, PyTorch}
%\location{Repo: \href{https://github.com/CaterinaBi/udacity-ai-programming}{udacity-ai-programming}}
%\begin{tightemize}
%\item Python programme that utilises an image classifier trained on ImageNet to identify dogs from non dogs, and dog breeds.
%\item Determines which Pytorch classification algorithm among AlexNet, VGG, and ResNet performs best at the task, and their runtime. Spoiler alert: it's ResNet!
%\end{tightemize}
%\sectionsep

%%%%%%%%%%%%%%%%%%%%%%%%%%%%%%%%%%%%%%
%
%     COLUMN TWO
%
%%%%%%%%%%%%%%%%%%%%%%%%%%%%%%%%%%%%%%

\end{minipage} 
\hfill
\begin{minipage}[t]{0.25\textwidth} 

%%%%%%%%%%%%%%%%%%%%%%%%%%%%%%%%%%%%%%
%     SKILLS
%%%%%%%%%%%%%%%%%%%%%%%%%%%%%%%%%%%%%%

\section{Skills}
\subsection{Programming}
\sectionsep
\location{Intermediate:} 
Python (2y) \textbullet{} LaTeX (5y) \\
\sectionsep
\location{Familiar:}
Java (>1y) \textbullet{}  HTML (<1y) \\
\sectionsep
%\sectionsep
\subsection{Libraries/Frameworks}
\sectionsep
OpenCV \textbullet{} TensorFlow \textbullet{} Selenium\\
PyTorch \textbullet{} pandas \textbullet{} MatPlotLib \\
NLTK \textbullet{} spaCy \textbullet{} seaborn \textbullet{} gensim 
\sectionsep
%\sectionsep
\subsection{Tools/Platforms}
\sectionsep
Git \textbullet{} GitHub \textbullet{} Docker \\
\sectionsep
%\sectionsep
\subsection{Linguistic Research}
\sectionsep
Syntax, Documentation, Data modelling, IPA, Praat.
\sectionsep

%%%%%%%%%%%%%%%%%%%%%%%%%%%%%%%%%%%%%%
%     EDUCATION
%%%%%%%%%%%%%%%%%%%%%%%%%%%%%%%%%%%%%%

\section{Education} 

%\subsection{MSc in Computer Science}
%\descript{University of York}
%\location{Nov 2021 - Present (on a leave of absence)}

%\sectionsep

\subsection{PhD in Linguistics}
\descript{Université de Genève}
\location{December 2015 - May 2019}

\sectionsep

\subsection{MA in Science of Language}
\descript{Università Ca' Foscari Venezia}
\location{October 2011 - June 2013}

\sectionsep

\subsection{BA in Science of Language}
\descript{Università Ca' Foscari}
\location{September 2008 - October 2011}

% %%%%%%%%%%%%%%%%%%%%%%%%%%%%%%%%%%%%%%
% %     CERTIFICATES
% %%%%%%%%%%%%%%%%%%%%%%%%%%%%%%%%%%%%%%
\section{Certificates}

\descript{\textbf{AiCore}}
\location{September 2022 - November 2022}
Software Engineer Certificate.

%\sectionsep

%\descript{\textbf{Udacity}}
%\location{March 2022 - October 2022}
%Programming in Python for AI.

%\sectionsep

%\descript{DataCamp}
%\location{October 2022 - Present}
%NLP in Python Skill Track.

\sectionsep

\descript{AiCore}
\location{May 2022 - September 2022}
Python Programmer Certificate.

\sectionsep

\descript{\textbf{Université de Genève}}
\location{September 2015 - December 2017}
Certificat de spécialisation en linguistique.
\sectionsep

%%%%%%%%%%%%%%%%%%%%%%%%%%%%%%%%%%%%%%
%     AWARDS
%%%%%%%%%%%%%%%%%%%%%%%%%%%%%%%%%%%%%%

\section{Languages} 

\begin{tabular}{ll}
Italian	     & native\\
French	     & near-native\\
English	     & proficient\\
Spanish      & conversational\\
%\\
\end{tabular}
\sectionsep

%%%%%%%%%%%%%%%%%%%%%%%%%%%%%%%%%%%%%%
%     CURIOSITIES
%%%%%%%%%%%%%%%%%%%%%%%%%%%%%%%%%%%%%%

\section{Curiosities}
I'm a retired pro volleyball player.

%%%%%%%%%%%%%%%%%%%%%%%%%%%%%%%%%%%%%%
%     COURSEWORK
%%%%%%%%%%%%%%%%%%%%%%%%%%%%%%%%%%%%%%

% \section{Coursework}

% \subsection{Graduate}
% Graduate Algorithms \textbullet{}\\ 
% Advanced Computer Architecture \textbullet{}\\ 
% Operating Systems \textbullet{}\\ 
% Artificial Intelligence \textbullet{}\\
% Visualization For Scientific Data \\
% \sectionsep

% \subsection{Undergraduate}

% Database Management Systems \textbullet{}\\
% Object Oriented Analysis and Design \textbullet{}\\
% Artificial Intelligence and Expert Systems \textbullet{}\\
% Scripting Languages and Web Tech \textbullet{}\\
% Software Engineering \\

\end{minipage} 
\end{document}  \documentclass[]{article}