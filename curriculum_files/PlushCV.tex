%%%%%%%%%%%%%%%%%%%%%%%%%%%%%%%%%%%%%%%
% PlushCV - One Page Two Column Resume
% LaTeX Template
% Version 1.0 (11/28/2021)
%
% Author:
% Shubham Mazumder (http://mazumder.me)
%
% Hacked together from:
% https://github.com/deedydas/Deedy-Resume
%
% IMPORTANT: THIS TEMPLATE NEEDS TO BE COMPILED WITH XeLaTeX
%
% 
%%%%%%%%%%%%%%%%%%%%%%%%%%%%%%%%%%%%%%
% 
% TODO:
% 1. Figure out a smoother way for the document to flow onto the next page.
% 3. Add more icon options 
% 4. Fix hacky left alignment on contact line
% 5. Remove Hacky fix for awkward extra vertical space
% 
%%%%%%%%%%%%%%%%%%%%%%%%%%%%%%%%%%%%%%
%
% CHANGELOG:
%
%%%%%%%%%%%%%%%%%%%%%%%%%%%%%%%%%%%%%%%
%
% Known Issues:
% 1. Overflows onto second page if any column's contents are more than the vertical limit.
%%%%%%%%%%%%%%%%%%%%%%%%%%%%%%%%%%%%%%
%%Icons:
%%Main: https://icons8.com/icons/carbon-copy
%%%%%%%%%%%%%%%%%%%%%%%%%%%%%%%%%%

\documentclass[]{plushcv}
\usepackage{fancyhdr}
\pagestyle{fancy}
\fancyhf{}
\begin{document}

%%%%%%%%%%%%%%%%%%%%%%%%%%%%%%%%%%%%%%
%
%     TITLE NAME
%
%%%%%%%%%%%%%%%%%%%%%%%%%%%%%%%%%%%%%%

\namesection{Caterina}{Bonan}{PhD in Linguistics | Junior NLP Engineer}{\contactline{\href{https://orcid.org/0000-0002-4808-6865}{OrcID}}{\href{https://github.com/CaterinaBi}{CaterinaBi}}{\href{https://www.linkedin.com/in/caterinabonan/}{LinkedIn}}{\href{mailto:caterina.bonan@outlook.com}{caterina.bonan@outlook.com}}{\href{tel:+447983818663}{07983818663}}}

% \namesection{Firstname}{Lastname}{Full Stack Software Engineer}{\contactline{\href{https://www.mazumder.me}{mazumder.me}}{\href{https://www.github.com/sansquoi}{sansquoi}}{\href{https://www.linkedin.com/mazumders}{mazumders}}{\href{mailto:shubham.mazumder@gmail.com}{first.last@email.com}}{\href{tel:+1999999999}{9999999999}}}

%%%%%%%%%%%%%%%%%%%%%%%%%%%%%%%%%%%%%%
%
%     COLUMN ONE
%
%%%%%%%%%%%%%%%%%%%%%%%%%%%%%%%%%%%%%%

\begin{minipage}[t]{0.70\textwidth} 


%%%%%%%%%%%%%%%%%%%%%%%%%%%%%%%%%%%%%%
%     EXPERIENCE
%%%%%%%%%%%%%%%%%%%%%%%%%%%%%%%%%%%%%%

\section{Selected experience}

\runsubsection{AiCore}
\descript{| Data and AI trainee}
\location{May 2022 – Present | London, UK}
%\vspace{\topsep} % Hacky fix for awkward extra vertical space
Training in AI \& Data Engineering using industry-standard tools. Working on multiple industry projects throughout.
\sectionsep
\sectionsep
\begin{tightemize}
\item Skills: Software engineering (Git \& GitHub, advanced Python, algorithms \& data structures); Data engineering (SQL, data lakes, data warehousing, web scraping);
Cloud Engineering (cloud computing, designing/building APIs, Docker, Apache Airflow, AWS Serverless Stack).
\end{tightemize}
\sectionsep

\runsubsection{University of Cambridge}
\descript{| Postdoctoral Researcher}
\location{September 2019 – Current | Cambridge, UK}
%\vspace{\topsep} % Hacky fix for awkward extra vertical space
Postdoctoral research on comparative Romance syntax. Worked on 30+ standard and non-standard varieties throughout. Published 8 peer-reviewed research articles, 2 edited volumes and 1 research monograph. Awarded a SNSF 'PostDoc.mobility' scholarship twice.
\sectionsep
\begin{tightemize}
\item Tools: Excel for data analysis and visualisation, Python, MatPlotLib, HF Transformers, SQL, LaTeX, VS code, Microsoft Word and Power Point.
\item Links to scientific writing samples: \href{https://periodicos.unb.br/index.php/cs/article/view/40559/33149}{\textbf{Cadernos}}, \href{https://ebooks.iospress.nl/doi/10.3233/SHTI220702}{\textbf{IOS Press}}, \href{https://doi.org/10.16995/glossa.5714}{\textbf{Glossa}}, \href{https://doi.org/10.5565/rev/isogloss.108}{\textbf{Isogloss}}.
\end{tightemize}
\sectionsep

\runsubsection{Université de Genève}
\descript{| Research Assistant }
\location{September 2015 – August 2019 | Geneva, CH}
Fully-funded doctoral research as part of the SNSF-funded 'Wh-in situ in French interrogatives: Prosody and Syntax' research group. Worked on standard and non-standard French, and 50+ northern Italian varieties. Published 3 research articles, 1 co-edited volume, and a PhD dissertation.
\vspace*{-2mm}
\begin{tightemize}
\item Tools: Excel for data analysis and visualisation, Python, Praat, LaTeX, Microsoft Word and Power Point.
\end{tightemize}
\sectionsep

%\runsubsection{ELSE: Conseil, Création, Édition}
%\descript{| Translator (IT>FR) }
%\location{March 2014 – May 2014 | Paris, FR}
%Translation of book 'Grands écrivains: Les auteurs célèbres vus par de grands photographes' (Éditions du Chêne, 522 pp.; ISBN 978-2-8123-1085-0).
%\sectionsep


%%%%%%%%%%%%%%%%%%%%%%%%%%%%%%%%%%%%%%
%     Projects
%%%%%%%%%%%%%%%%%%%%%%%%%%%%%%%%%%%%%%

\section{Experience through projects}

\runsubsection{FB MARKETPLACE RECOMMENDATION RANKING SYSTEM}
\descript{| Python, PyTorch, AWS, EKS.}
\location{Repo: \href{https://github.com/CaterinaBi/aicore-recommendation-ranking-system}{aicore-recommendation-ranking-system}}
\begin{tightemize}
\item Currently training a model that works with images and text to make recommendations to potential buyers based on their demographic information;
\item Will host the app on EKS to orchestrate the containers that maintain/retrain the model.
\end{tightemize}
\sectionsep

\runsubsection{USING NLP TO CREATE SPOKEN CORPORA}
\descript{| Python, Selenium, NLTK, PyTorch.}
\location{Repo: \href{https://github.com/CaterinaBi/spoken-french-scraping-pipeline}{spoken-french-scraping-pipeline}}
\begin{tightemize}
\item Currently developing a module that scrapes text and audio data and saves them locally.
\item The audio data will be transcribed, tokenised and tagged using NLTK and PyTorch. 
\end{tightemize}
\sectionsep

\runsubsection{WEB SCRAPING PIPELINE}
\descript{| Python, Selenium, Docker, GitHub Actions.}
\location{Repo: \href{https://github.com/CaterinaBi/aicore-web-scraping-pipeline}{aicore-web-scraping-pipeline}}
\begin{tightemize}
\item Developed a module that scrapes data to upsert information to a database;
\item Pushed an image of the containerised application in a CI/CD pipeline that performs unit testing and deploys the image to an EC2 instance.
\end{tightemize}
\sectionsep

%\runsubsection{CHATBOT WITH DEEP NLP}
%\descript{| Python, Seq2Seq, TensorFlow.}
%\location{Repo: \href{https://github.com/CaterinaBi/find-a-holiday-chatbot}{find-a-holiday-chatbot}}
%\begin{tightemize}
%\item Chatbot that utilises a Seq2Seq architecture and TensorFlow to help choose the perfect holiday destination.
%\end{tightemize}
%\sectionsep

\runsubsection{IMAGE CLASSIFIER FOR FLOWER RECOGNITION}
\descript{| Python, TensorFlow.}
\location{Repo: \href{https://github.com/CaterinaBi/udacity-image-classifier}{udacity-image-classifier}}
\begin{tightemize}
\item Python image classification application that runs from the command line. 
\item The application trains a deep learning model on the 102 Category Flower Dataset, and uses it to classify images of flowers.
\end{tightemize}
\sectionsep

\runsubsection{COMPUTER VISION APPLICATION}
\descript{| Python, Teachable Machine, OpenCV}
\location{Repo: \href{https://github.com/CaterinaBi/aicore-computer-vision}{aicore-computer-vision}}
\begin{tightemize}
\item Trained a computer vision model and used Tensorflow to detect whether Rock, Paper or Scissors is shown to the camera in real-time and with a high accuracy
\item Used the OpenCV library to access the webcam and play Rock Paper Scissors with the computer using the image from the camera
\end{tightemize}
\sectionsep

%\runsubsection{PRE-TRAINED IMAGE CLASSIFIER FOR DOG BREEDS}
%\descript{| Python, PyTorch}
%\location{Repo: \href{https://github.com/CaterinaBi/udacity-ai-programming}{udacity-ai-programming}}
%\begin{tightemize}
%\item Python programme that utilises an image classifier trained on ImageNet to identify dogs from non dogs, and dog breeds.
%\item Determines which Pytorch classification algorithm among AlexNet, VGG, and ResNet performs best at the task, and their runtime. Spoiler alert: it's ResNet!
%\end{tightemize}
%\sectionsep

%%%%%%%%%%%%%%%%%%%%%%%%%%%%%%%%%%%%%%
%
%     COLUMN TWO
%
%%%%%%%%%%%%%%%%%%%%%%%%%%%%%%%%%%%%%%

\end{minipage} 
\hfill
\begin{minipage}[t]{0.25\textwidth} 

%%%%%%%%%%%%%%%%%%%%%%%%%%%%%%%%%%%%%%
%     SKILLS
%%%%%%%%%%%%%%%%%%%%%%%%%%%%%%%%%%%%%%

\section{Skills}
\subsection{Programming}
\sectionsep
\location{Intermediate:} 
Python (>2y) \textbullet{} LaTeX (>5y)  \\
\sectionsep
\location{Familiar:}
Java (>1y) \textbullet{}  HTML (<1y) \\
\sectionsep
%\sectionsep
\subsection{Libraries/Frameworks}
\sectionsep
OpenCV \textbullet{} TensorFlow \textbullet{} Selenium\\
PyTorch \textbullet{} pandas \textbullet{} MatPlotLib \\
NLTK \textbullet{} SpeechRecognition \textbullet{} \\ 
PyDub \textbullet{} spaCy \textbullet{} polyglot
\sectionsep
%\sectionsep
\subsection{Tools/Platforms}
\sectionsep
Git \textbullet{} GitHub \textbullet{} Docker \textbullet{} SQL \\
\sectionsep
%\sectionsep
\subsection{Linguistic Research}
\sectionsep
Syntax, Documentation, Data modelling, IPA, Praat.
\sectionsep

%%%%%%%%%%%%%%%%%%%%%%%%%%%%%%%%%%%%%%
%     EDUCATION
%%%%%%%%%%%%%%%%%%%%%%%%%%%%%%%%%%%%%%

\section{Education} 

%\subsection{MSc in Computer Science}
%\descript{University of York}
%\location{Nov 2021 - Present (on a leave of absence)}

%\sectionsep

\subsection{PhD in Linguistics}
\descript{Université de Genève}
\location{December 2015 - May 2019}

\sectionsep

\subsection{MA in Science of Language}
\descript{Università Ca' Foscari Venezia}
\location{October 2011 - June 2013}

\sectionsep

\subsection{BA in Science of Language}
\descript{Università Ca' Foscari}
\location{September 2008 - October 2011}

% %%%%%%%%%%%%%%%%%%%%%%%%%%%%%%%%%%%%%%
% %     CERTIFICATES
% %%%%%%%%%%%%%%%%%%%%%%%%%%%%%%%%%%%%%%
\section{Certificates}

\descript{\textbf{AiCore}}
\location{September 2022 - November 2022}
Software Engineer Certificate.

\sectionsep

\descript{\textbf{Udacity}}
\location{March 2022 - October 2022}
Programming in Python for AI.

\sectionsep

\descript{DataCamp}
\location{October 2022 - Present}
NLP in Python Skill Track.

\sectionsep

\descript{AiCore}
\location{May 2022 - September 2022}
Python Programmer Certificate.

\sectionsep

\descript{Université de Genève}
\location{December 2015 - December 2017}
Specialisation in Linguistics.
\sectionsep

%%%%%%%%%%%%%%%%%%%%%%%%%%%%%%%%%%%%%%
%     AWARDS
%%%%%%%%%%%%%%%%%%%%%%%%%%%%%%%%%%%%%%

%\section{Awards} 
%\begin{tabular}{rll}
%2020	     & SNSF & Postdoctoral mobility\\funds (100.000+chf)\\
%2018	     & SNSF & Postdoctoral mobility\\funds (100.000+chf)\\
%2015	     & SNSF  & Doctoral funds\\(200.000+chf)\\
%\\
%\end{tabular}
%\sectionsep

%%%%%%%%%%%%%%%%%%%%%%%%%%%%%%%%%%%%%%
%     CURIOSITIES
%%%%%%%%%%%%%%%%%%%%%%%%%%%%%%%%%%%%%%

\section{Curiosities}
I'm a retired pro volleyball player, and since 2015 I have attracted research funds for more than 400.000chf (about £375.000).

%%%%%%%%%%%%%%%%%%%%%%%%%%%%%%%%%%%%%%
%     COURSEWORK
%%%%%%%%%%%%%%%%%%%%%%%%%%%%%%%%%%%%%%

% \section{Coursework}

% \subsection{Graduate}
% Graduate Algorithms \textbullet{}\\ 
% Advanced Computer Architecture \textbullet{}\\ 
% Operating Systems \textbullet{}\\ 
% Artificial Intelligence \textbullet{}\\
% Visualization For Scientific Data \\
% \sectionsep

% \subsection{Undergraduate}

% Database Management Systems \textbullet{}\\
% Object Oriented Analysis and Design \textbullet{}\\
% Artificial Intelligence and Expert Systems \textbullet{}\\
% Scripting Languages and Web Tech \textbullet{}\\
% Software Engineering \\

\end{minipage} 
\end{document}  \documentclass[]{article}